\documentclass{article}

\usepackage[margin=2.5cm]{geometry}
\usepackage{amsmath, amssymb}

\usepackage{float}
\floatplacement{table}{htbp}

\usepackage{graphicx}
\usepackage{siunitx}

\usepackage{booktabs}
\usepackage{scrhack}
\usepackage[german]{hyperref}


\begin{document}
\section*{Aufgabe 1}
\begin{figure}[H]
	\centering
	\includegraphics{aufgabe1/aufgabe1-plot.pdf}
\end{figure}
Somit lautet die Federkonstante
\[
	k = 1.752 \frac{\si{\kg}}{\si{\second^2}}.
\]

\newpage
\section*{Aufgabe 2}
\subsection*{a)}
\begin{table}
	\centering
	\begin{tabular}{c c c}
		\toprule
		$g/\si{\mm}$ & $b/\si{\mm}$ & 
		$f/\si{\mm}$ \\
		\midrule
		60	& 285	& 49.565217 \\
		80	& 142	& 51.171171 \\
		100	& 117	& 53.917051 \\
		110	& 85	& 47.948718 \\
		120	& 86	& 50.097087 \\
		125	& 82	& 49.516908 \\
		\bottomrule
	\end{tabular}
\end{table}

Man erhält Mittelwert, Standartabweichung und Messunsicherheit lauten
dann von
\begin{align*}
	m &= \frac{1}{n} \sum_k x_k =  50.36\si\mm
	\\
	\sigma &= 1.85 \si\mm
	\\
	u &= \frac{1}{\sqrt{n}} \sigma 
	= \frac{1}{\sqrt{6}} \cdot 1.85\si\mm
	= 0.755 \si\mm
\end{align*}

\subsection*{b)}

\begin{figure}[H]
	\centering
	\includegraphics{aufgabe2/aufgabe2-plot.pdf}
\end{figure}

Die lineare Regression liefert die Gleichung
\begin{equation}
	F := \frac{1}{f} = m\cdot x + b
	\label{eqn:geradengleichung}
\end{equation}
mit den Werten
\[
	m = -1.01 \quad b = \frac{0.01998}{\si\mm}
\]
Setzt man in \autoref{eqn:geradengleichung} $x=0$ erhält man
\[
	\frac{1}{f} = b \Leftrightarrow
	f = \frac{1}{b} \approx 50.043620 \si\mm
\]

\subsection*{c)}
Die zwei Werte liegen sehr nah beieinander mit einer Differenz
\[
	\Delta m = m_1 - m_2 \approx 0.32 \si\mm
\]
Welcher kleiner als die Messungenauigkeit $u$ in a) ist, weswegen
beide Methoden als gleich gut zu bewerten sind.

\section*{Aufgabe 3}
\begin{table}
	\centering
	\begin{tabular}{c c c}
		\toprule
		$d/\si{\cm}$ & $N \cdot \si{\s}$ & 
		$\Delta N/\si{\s}$ \\
		\midrule
		0.1  & 126.083333  & 11.228684 \\
		0.2  & 115.116667  & 10.729244 \\
		0.3  & 103.566667  & 10.176771 \\
		0.4  &  92.183333  &  9.601215 \\
		0.5  &  82.366667  &  9.075608 \\
		1.0  &  44.200000  &  6.648308 \\
		1.2  &  36.100000  &  6.008328 \\
		1.5  &  24.433333  &  4.943009 \\
		2.0  &  16.166667  &  4.020779 \\
		3.0  &   5.550000  &  2.355844 \\
		4.0  &   2.116667  &  1.454877 \\
		5.0  &   0.800000  &  0.894427 \\
		\bottomrule
	\end{tabular}
\end{table}
Der Curve-Fit ergab den Wert
\[
	\mu = 1.13\si{\cm}.
\]

\newpage
Ein linearer und ein halblogarithmischer Plot sind in \autoref{fig:geile-plots} zu sehen
\begin{figure}[H]
	\centering
	\includegraphics{aufgabe3/aufgabe3-plot.pdf}
	\caption{Zwei geile Plots.}
	\label{fig:geile-plots}
\end{figure}

\end{document}
